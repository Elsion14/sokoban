\documentclass{beamer}
\usetheme{PaloAlto}
\definecolor{fond}{rgb}{0.123,0.364,0.48}
\setbeamercolor{structure}{fg=fond,bg=fond!40}
\usepackage[T1]{fontenc}
\usepackage[utf8]{inputenc}
 
\title{Sokoban}
\author{Chagneux, Leblond, Beauchamp, Mori}
 \date{Mars-2018}
\begin{document} 
\frame{\titlepage}
\begin{frame}
 \tableofcontents
\end{frame}
\section{Objectifs du projet}
\begin{frame}
Objectifs du projet
\end{frame}

\begin{frame}
Description
\end{frame}
\section{Éléments techniques}
\begin{frame}
Éléments techniques
\end{frame}

\section{Architecture du projet}
\begin{frame}
\begin{columns}
\begin{column}{4cm}
\includegraphics[scale=0.2]{Sokoban.png}
\end{column}

\begin{column}{4cm}
\includegraphics[scale=0.2]{Sokoban.png}
\end{column}

\begin{column}{4cm}
\includegraphics[scale=0.2]{Sokoban.png}
\end{column}
\end{columns}
\end{frame}

\section{Expérimentations et usages}
\begin{frame}
\begin{block}{Théorème}
C'est un théorème
\end{block}
\pause
\begin{exampleblock}{Exemple 1}
Exemple
\end{exampleblock}
\end{frame}

\section{Conclusion}
\begin{frame}
\begin{alertblock}{Observation 1}
Ceci est une observation
\end{alertblock}
\pause
\begin{exampleblock}{Exemple 2}
Exemple
\end{exampleblock}
\end{frame}
\end{document}